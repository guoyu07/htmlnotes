\part{HTML Style}

很多标签都可以用来改变文本的外观,并为文本关联其隐藏的含义。总地来说,这些标签可以分成两类:基于内容的样式(content-based style)和物理样式(physical style)。

同时,W3C 为级联样式表(CSS)指定的标准现在已被绝大多数浏览器所支持\footnote{自HTML4.0开始,所有的格式化代码均可移出 HTML 文档,然后移入一个独立的外部样式表。},它提供了一种允许作者控制文档文本外观和布局的更为全面的方法。

\chapter{Content-based Style}


基于内容的样式标签会告诉浏览器它所包含的文本具有特定的含义、上下文或者用法。然后浏览器就会把与该含义、上下文或者用法一致的格式应用在文本上。请注意这里面的区别。基于内容的标签赋予含义,而不是格式化。因此,它们对于自动处理来说非常重要;计算机并不关心文档的外观如何。

因为字体样式是通过语义线索来指定的,因此浏览器可以为用户选择一种合适的显示样式。由于不同地点的样式各种各样,所以使用基于内容的样式可以帮助用户确保自己的文档对广大范围的读者来说都是有意义的。这一点在专门供那些盲人和残疾人所使用的浏览器上显得尤其重要,因为他们的显示选项可能和我们传统的文本根本不同,或者在某方面具有非常大的局限性。

当前的 HTML 和 XHTML 标准并没有为每一个基于内容的标签都定义一种格式;它们仅仅规定必须用与文档中普通文本不同的方式来显示基于内容的样式。标准甚至没有要求这些基于内容的样式彼此之间都要用不同的方式显示。在实际应用中,你可能会发现很多这样的标签和传统的印刷有着非常明显的关系,它们有着相似的含义和显示样式,而且在多数浏览器中都以相同的样式和字体来显示。

使用 HTML/XHTML 基于内容的样式标签时要遵从一些规则,因为仅仅是简单地想想文本该如何显示,而不必知道这些文本的含义是什么,是十分容易的。一旦你开始使用基于内容的样式之后,文档将会更加一致,而且可以更好地帮助执行自动搜索和内容编辑。这些标签是:

\begin{compactitem}
\item <abbr>
\item <acronym>
\item <cite>
\item <code>
\item <dfn>
\item <em>
\item <kbd>
\item <samp>
\item <strong>
\item <var>
\end{compactitem}


\chapter{Physical Style}


在讨论基于内容的样式标签时,我们经常用到“意图”这个词。这是因为由标签传达的含义比浏览器显示文本的方式更为重要。然而,在某些情况下,可能是出于合法性或者版权等方面的原因的考虑,用户希望文本以某种特殊的方式来显示(例如斜体或加粗)。在这种情况下,就可以对文本使用物理样式。

虽然其他文字处理系统的趋势是精确地控制样式和外观,但是在使用 HTML 或 XHTML 时,除非极少情况下,都应该避免使用物理标签。应当尽可能地向浏览器提供上下文信息,并使用基于内容的样式。尽管现在浏览器不过是以斜体或者粗体字来显示这些文本,但是将来的浏览器和各种文档生成工具可能会以非常有创见的方式来利用这些基于内容的样式。


当前的 HTML/XHTML 标准一共提供了 9 种物理样式,包括粗体(bold)、斜体(italic)、等宽(monospaced)、下划线(underlined)、删除线(strikethrough)、放大(larger)、缩小(smaller)、上标(superscripted)和下标(subscripted)文本。这些标签\footnote{记住这些物理样式标签对紧接的文本产生的强烈效果。要实现在整个文档范围内对文本显示的全面控制,应该使用样式表。}是:

\begin{compactitem}
\item <b>
\item <big>
\item <i>
\item <s>
\item <small>
\item <strike>
\item <sub>
\item <sup>
\item <tt>
\end{compactitem}

\chapter{HTML CSS}

当浏览器读到一个样式表,它就会按照这个样式表来对文档进行格式化,从而达到精确地控制页面设置与表现形式(边距,悬浮,对齐,宽度,高度等)的目的。有以下三种方式来插入样式表:

\begin{compactitem}
\item 外部样式表
\item 内部样式表
\item 内联样式
\end{compactitem}

另外,float属性用以定义元素的漂浮方式:靠左还是靠右。下例展示了该属性的用法:

\begin{lstlisting}[language=HTML]
<img src="x.jpg" alt="pic" style= "float:left;" />
\end{lstlisting}

而可以通过设置position属性,把元素精确地控制在页面的某个位置。

\begin{lstlisting}[language=HTML]
<img src="x.jpg" alt="pic" style="position:absolute;bottom:5px;right:5px;" />
\end{lstlisting}

在这里,图像被放置在浏览器中位于距底端50象素、距右边界10象素的位置,而实际上用户可以把它放在他们所期望的任何位置上。

\section{External Style Sheet}


当样式需要被应用到很多页面的时候,外部样式表将是理想的选择。使用外部样式表,用户就可以通过更改一个文件来改变整个站点的外观。

要将CSS嵌入文档,用户只需通过标签<style type="text/css">告诉浏览器该段为CSS。

\begin{lstlisting}[language=HTML]
<head>
	<link rel="stylesheet" type="text/css" href="mystyle.css">
	...
	<title>...</title>
</head>
\end{lstlisting}




\section{Internal Style Sheet}


当单个文件需要特别样式时,就可以使用内部样式表。用户可以在 head 部分通过 <style> 标签定义内部样式表。


\begin{lstlisting}[language=HTML]
<head>
	<style type="text/css">
		body {background-color: red}
		p {margin-left: 20px}
	</style>
	...
	<title>...</title>
</head>
\end{lstlisting}



\section{Inline Styles}

当特殊的样式需要应用到个别元素时,就可以使用内联样式。 使用内联样式的方法是在相关的标签中使用样式属性。样式属性可以包含任何 CSS 属性。以下实例显示出如何改变段落的颜色和左外边距。

\begin{lstlisting}[language=HTML]
<p style="color: red; margin-left: 20px">
This is a paragraph
</p>
\end{lstlisting}




\chapter{Span and div}


In HTML, the span and div elements are used for generic organizational or stylistic applications, typically when extant meaningful elements have exhausted their purpose.


span和div元素用于组织和结构化文档,并经常联合class和id属性一起使用。


Most HTML elements signify the specific meaning of their content – i.e. the element describes, and can be made to function according to, the type of data contained within. For example, a p element should contain a paragraph of text, while an h1 element should contain the highest-level heading of the page or section; user agents should distinguish them accordingly. span and div signify no specific meaning besides the generic grouping of content, and are therefore more appropriate for creating organization or stylistic additions without signifying superfluous meaning.

可以通过 <div> 和 <span> 将 HTML 元素组合起来,但是在实际应用中,都是使用级联样式表(Cascading Style Sheets,简称CSS)为网站设计页面布局。

CSS是HTML的搭档。在编码过程中,它们发挥不同的作用:HTML负责网页的具体内容(结构),而CSS则修饰网页的表现形式(布局)。

CSS有一个优越的特性,即它可以对页面布局进行集中管理。也就是说,用户不必在每个标签里都使用style属性;相反,可以只声明一次,浏览器便会按相应的页面布局效果来显示文本。

我们在HTML文档的头部(head)引入的CSS,它将应用于整个页面。通过把CSS文档独立出来,用户就可以同时对多个页面的布局进行集中管理。如果用户要对一个大型网站上的大量网页作字体类型或大小的修改,那么这个方法绝对是最佳选择。

\section{span}

span元素可以说是一种中性元素,因为它不对文档本身添加任何东西。但是与CSS结合使用的话,span可以对文档中的部分文本增添视觉效果。


\section{div}

span的使用局限在一个块元素内,而div可以被用来组织一个或多个块元素。除去这点区别,div和span在组织元素方面相差无几。




\section{Layout}


网页布局对改善网站的外观非常重要,用户要慎重设计网页布局。

大多数网站会把内容安排到多个列中(就像杂志或报纸那样)。

可以使用 <div> 或者 <table> 元素来创建多列\footnote{即使可以使用 HTML 表格来创建漂亮的布局,但设计表格的目的是呈现表格化数据 - 表格不是布局工具!}。CSS 用于对元素进行定位,或者为页面创建背景以及色彩丰富的外观。

下面的例子使用五个 div 元素来创建多列布局:

\begin{lstlisting}[language=HTML]
<!DOCTYPE html>
<html>
<head>
	<style type="text/css">
	div#container{width:500px}
	div#header {background-color:#99bbbb;}
	div#menu {background-color:#ffff99; height:200px; width:100px; float:left;}
	div#content {background-color:#EEEEEE; height:200px; width:400px; float:left;}
	div#footer {background-color:#99bbbb; clear:both; text-align:center;}
	h1 {margin-bottom:0;}
	h2 {margin-bottom:0; font-size:14px;}
	ul {margin:0;}
	li {list-style:none;}
	</style>
</head>

<body>
<div id="container">
	<div id="header">
		<h1>Main Title of Web Page</h1>
	</div>
	<div id="menu">
	<h2>Menu</h2>
	<ul>
		<li>HTML</li>
		<li>CSS</li>
		<li>JavaScript</li>
	</ul>
	</div>
	<div id="content">Content goes here</div>
	<div id="footer">Copyright</div>
</div>
</body>
</html>
\end{lstlisting}

使用 CSS 最大的好处是,如果把 CSS 代码存放到外部样式表中,那么站点会更易于维护。用户通过编辑单一的文件,就可以改变所有页面的布局。

由于创建高级的布局非常耗时,使用模板是一个快速的选项。通过搜索引擎可以找到很多免费的网站模板(可以使用这些预先构建好的网站布局,并优化它们)。


\begin{table}
\centering
\caption{HTML 布局标签}
\begin{tabular}{|l|l|}
\hline
标签		&描述\\
\hline
<div>	&定义文档中的分区或节(division/section)。\\
\hline
<span>	&定义 span,用来组合文档中的行内元素。\\
\hline
\end{tabular}
\end{table}


\section{Differences and default behavior}


There are multiple differences between div and span. The most notable difference is how the elements are displayed. In standard HTML, a div is a block-level element whereas a span is an inline element. The div block visually isolates a section of a document on the page, and may contain other block-level components. The span element contains a piece of information inline with the surrounding content, and may only contain other inline-level components. In practice, the default display of the elements can be changed by the use of Cascading Style Sheets (CSS), however the permitted contents of each element may not be changed. For example, regardless of CSS, a span element may not contain block-level children.



\section{ Practical usage}

span and div elements are used purely to imply a logical grouping of enclosed elements.

There are three main reasons to use span and div tags with class or id attributes:





\subsection{Styling with CSS}

Perhaps the most common use of <span> and <div> elements is to carry class or id attributes in conjunction with CSS to apply layout, typographic, color, and other presentation attributes to parts of the content. CSS does not just apply to visual styling: when spoken out loud by a voice browser, CSS styling can affect speech-rate, stress, richness and even position within a stereophonic image.

For these reasons, and for compatibility with the concepts of the semantic web, discussed below, attributes attached to elements within any HTML should describe their semantic purpose, rather than merely their intended display properties in one particular medium. For example, the HTML in <span class="red-bold">password too short</span> is semantically weak, whereas <em class="warning">password too short</em> uses an em element to signify emphasis, and uses a more appropriate class name. By the correct use of CSS, 'warnings' may be rendered in a red, bold font on a screen, but when printed out they may be omitted, as by then it is too late to do anything about them. Perhaps when spoken they should be given extra stress, and a small reduction in speech-rate. The second example is semantically correct markup, rather than merely presentational.





\subsection{Semantic clarity}

This kind of grouping and labeling of parts of the page content might be introduced purely to make the page more semantically meaningful in general terms. It is impossible to say how and in what ways the World Wide Web will develop in years and decades to come. Web pages designed today may still be in use when information systems that we cannot yet imagine are trawling, processing, and classifying the web. Even today's search engines such as Google and others use proprietary information processing algorithms of considerable complexity.

For some years, the World Wide Web Consortium (W3C) has been running a major Semantic Web project designed to make the whole web increasingly useful and meaningful to today's and the future's information systems.

The microformats movement is an attempt to build an idea of semantic classes. For example, microformats-aware software might automatically find an element like <span class="tel">123-456-7890</span> and allow for automatic dialing of the telephone number.



\subsection{Access from code}

Once the HTML or XHTML markup is delivered to a page-visitor's client browser, there is a chance that client-side code will need to navigate the internal structure (or Document Object Model) of the web page. The most common reason for this is that the page is delivered with client-side JavaScript that will produce on-going dynamic behavior after the page is rendered. For example, if rolling the mouse over a 'Buy now' link is meant to make the price, elsewhere on the page, become emphasized, JavaScript code can do this, but JavaScript needs to identify the price element, wherever it is in the markup, in order to affect it. The following markup would suffice: <div id="price">\$45.99</div>. Another example is the Ajax programming technique, where, for example, clicking a hypertext link may cause JavaScript code to retrieve the text for a new price quotation to display in place of the current one within the page, without re-loading the whole page. When the new text arrives back from the server, the JavaScript must identify the exact region on the page to replace with the new information.

Less common, but just as important examples of code gaining access to final web pages, and having to use span and div elements' class or id attributes to navigate within the page include the use of automatic testing tools. On dynamically generated HTML, this may include the use of automatic page testing tools such as HttpUnit, a member of the xUnit family, and load or stress testing tools such as Apache JMeter when applied to form-driven web sites.


\section{Overuse}

The judicious use of div and span is a vital part of HTML and XHTML markup. However, they are sometimes overused.

For example, when structurally and semantically a series of items need an outer, containing element and then further containers for each item, then there are various list structures available in HTML, one of which may be preferable to a homemade mixture of div and span elements.[3]
For example, these code below

\begin{lstlisting}[language=HTML]
<ul class="menu">
  <li>Main page</li>
  <li>Contents</li>
  <li>Help</li>
</ul>
\end{lstlisting}

...is usually preferable to this:

\begin{lstlisting}[language=HTML]
<div class="menu">
  <span>Main page</span>
  <span>Contents</span>
  <span>Help</span>
</div>
\end{lstlisting}


Other examples of the semantic use of HTML rather than div and span elements include the use of fieldset elements to divide up a web form, the use of legend elements to identify such divisions and the use of label to identify form input elements rather than div, span or table elements used for such purposes.

HTML5 introduces new elements; a few examples include the header, footer, nav and figure elements.



































































