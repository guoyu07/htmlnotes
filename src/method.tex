\part{HTTP methods}


设计超文本传输协议(HTTP)的目的是保证客户机与服务器之间的通信,它的工作方式是客户机与服务器之间的请求-应答协议。

Web浏览器可能是客户端,而计算机上的网络应用程序也可能作为服务器端。客户端(浏览器)向服务器提交 HTTP 请求;服务器向客户端返回响应。响应包含关于请求的状态信息以及可能被请求的内容。


在客户机和服务器之间进行请求-响应时,两种最常被用到的方法是:GET 和 POST。

\begin{compactitem}
\item GET - 从指定的资源请求数据。
\item POST - 向指定的资源提交要被处理的数据。
\end{compactitem}


\chapter{GET}

请注意,查询字符串(名称/值对)是在 GET 请求的 URL 中发送的:

\begin{lstlisting}[language=HTML]
    /test/demo_form.asp?name1=value1&name2=value2
\end{lstlisting}

有关 GET 请求的其他一些注释:

\begin{compactitem}
\item GET 请求可被缓存
\item GET 请求保留在浏览器历史记录中
\item GET 请求可被收藏为书签
\item GET 请求不应在处理敏感数据时使用
\item GET 请求有长度限制
\item GET 请求只应当用于取回数据
\end{compactitem}


\chapter{POST}

请注意,查询字符串(名称/值对)是在 POST 请求的 HTTP 消息主体中发送的:

\begin{lstlisting}[language=bash]
POST /test/demo_form.asp HTTP/1.1
Host: w3schools.com
name1=value1&name2=value2
\end{lstlisting}

有关 POST 请求的其他一些注释:

\begin{compactitem}
\item POST 请求不会被缓存
\item POST 请求不会保留在浏览器历史记录中
\item POST 不能被收藏为书签
\item POST 请求对数据长度没有要求
\end{compactitem}

下面的表格比较了两种 HTTP 方法:GET 和 POST。

\begin{longtable}{|p{60pt}|p{150pt}|p{150pt}|}
%head
\multicolumn{3}{r}{...}
\tabularnewline\hline
		& GET 	& POST
\endhead
%endhead

%firsthead
\hline
		& GET 	& POST
\endfirsthead
%endfirsthead

%foot
\multicolumn{3}{r}{...}
\endfoot
%endfoot

%lastfoot
\endlastfoot
%endlastfoot
\hline
后退按钮/刷新	&无害				&数据会被重新提交(浏览器应该告知用户数据会被重新提交)。\\
\hline
书签			&可收藏为书签		&不可收藏为书签\\
\hline
缓存			&能被缓存			&不能缓存\\
\hline
编码类型		&application/x-www-form-urlencoded	&application/x-www-form-urlencoded 或 multipart/form-data。为二进制数据使用多重编码。\\
\hline
历史	参数		&保留在浏览器历史中。&参数不会保存在浏览器历史中。\\
\hline
对数据长度的限制&是的。当发送数据时,GET 方法向 URL 添加数据;URL 的长度是受限制的(URL 的最大长度是 2048 个字符)。&无限制。\\
\hline
对数据类型的限制&只允许 ASCII 字符。	&没有限制。也允许二进制数据。\\
\hline
安全性	&与 POST 相比,GET 的安全性较差,因为所发送的数据是 URL 的一部分。\newline 在发送密码或其他敏感信息时绝不要使用 GET !& POST 比 GET 更安全,因为参数不会被保存在浏览器历史或 web 服务器日志中。\\
\hline
可见性	&数据在 URL 中对所有人都是可见的。&数据不会显示在 URL 中。\\
\hline


\end{longtable}



下面的表格列出了其他一些 HTTP 请求方法:

\begin{longtable}{|p{60pt}|p{300pt}|}
%head
\multicolumn{2}{r}{...}
\tabularnewline\hline
方法		&描述
\endhead
%endhead

%firsthead
\hline
方法		&描述
\endfirsthead
%endfirsthead

%foot
\multicolumn{2}{r}{...}
\endfoot
%endfoot

%lastfoot
\endlastfoot
%endlastfoot


\hline
HEAD		&与 GET 相同,但只返回 HTTP 报头,不返回文档主体。\\
\hline
PUT			&上传指定的 URI 表示。\\
\hline
DELETE		&删除指定资源。\\
\hline
OPTIONS		&返回服务器支持的 HTTP 方法。\\
\hline
CONNECT	&把请求连接转换到透明的 TCP/IP 通道。\\
\hline
\end{longtable}













































































