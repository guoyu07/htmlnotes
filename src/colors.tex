\part{HTML Colors}

颜色是通过对红、绿和蓝光的组合来显示的。

颜色由一个十六进制符号来定义,这个符号由红色、绿色和蓝色的值组成(RGB)。

每种颜色的最小值是0(十六进制:\#00)。最大值是255(十六进制:\#FF)。

这个表格给出了由三种颜色混合而成的具体效果:

\begin{table}[!h]
\centering
\caption{由三种颜色混合而成的具体效果示例}
\begin{tabular}{|p{100pt}|p{100pt}|p{100pt}|}
\hline
Color				&Color HEX		&Color RGB		\\
\hline
\cellcolor{Black}	&\#000000		&rgb(0,0,0)		\\
\hline
\cellcolor{Red}	&\#FF0000		&rgb(255,0,0)	\\
\hline
\cellcolor{Green}	&\#00FF00		&rgb(0,255,0)	\\
\hline
\cellcolor{Blue}	&\#0000FF		&rgb(0,0,255)	\\
\hline
\cellcolor{Yellow}	&\#FFFF00		&rgb(255,255,0)	\\
\hline
\cellcolor{Cyan}	&\#00FFFF		&rgb(0,255,255)	\\
\hline
\cellcolor{Magenta}	&\#FF00FF		&rgb(255,0,255)	\\
\hline
\cellcolor{Silver}	&\#C0C0C0		&rgb(192,192,192)\\
\hline
\cellcolor{White}	&\#FFFFFF		&rgb(255,255,255)\\
\hline
\end{tabular}
\end{table}

大多数的浏览器都支持颜色名集合,但要注意的是,HTML 和 CSS 颜色规范中定义了147种颜色名(17 种标准颜色\footnote{17 种标准色是 aqua, black, blue, fuchsia, gray, green, lime, maroon, navy, olive, orange, purple, red, silver, teal, white, yellow。}加 130 种其他颜色),但仅仅有 16 种颜色名被 W3C 的 HTML4.0 标准所支持。它们是:aqua, black, blue, fuchsia, gray, green, lime, maroon, navy, olive, purple, red, silver, teal, white, yellow。如果需要使用其它的颜色,需要使用十六进制的颜色值。

\begin{table}[!h]
\centering
\caption{使用十六进制的颜色值示例}
\begin{tabular}{|p{100pt}|p{100pt}|p{100pt}|}
\hline
Color				&Color HEX		&Color Name		\\
\hline
\cellcolor{AliceBlue}	&\#F0F8FF		&AliceBlue		\\
\hline
\cellcolor{AntiqueWhite}&\#FAEBD7		&AntiqueWhite	\\
\hline
\cellcolor{Aquamarine}	&\#7FFFD4		&Aquamarine	\\
\hline
\cellcolor{Black}		&\#000000		&Black	\\
\hline
\cellcolor{Blue}		&\#0000FF		&Blue	\\
\hline
\cellcolor{BlueViolet}	&\#8A2BE2		&BlueViolet	\\
\hline
\cellcolor{Brown}		&\#A52A2A		&Brown	\\
\hline
\end{tabular}
\end{table}


数年以前,当大多数计算机仅支持 256 种颜色的时候,一系列 216 种 Web 安全色作为 Web 标准被建议使用。其中的原因是,微软和 Mac 操作系统使用了 40 种不同的保留的固定系统颜色(双方大约各使用 20 种)。

我们不确定如今这么做的意义有多大,因为越来越多的计算机有能力处理数百万种颜色,不过做选择还是用户自己。


最初,216 跨平台 web 安全色被用来确保:当计算机使用 256 色调色板时,所有的计算机能够正确地显示所有的颜色。

下表提供了被大多数浏览器支持的颜色名。

重申一下,仅有 16 种颜色名被 W3C 的 HTML 4.0 标准支持,它们是:aqua、black、blue、fuchsia、gray、green、lime、maroon、navy、olive、purple、red、silver、teal、white、yellow。如果使用其它颜色的话,就应该使用十六进制的颜色值。


\begin{longtable}{|p{100pt}|p{100pt}|p{100pt}|}
\multicolumn{3}{r}{...}
\tabularnewline\hline
颜色名			& 十六进制颜色值	& 颜色				
\endhead
\hline
颜色名			& 十六进制颜色值	& 颜色		
\tabularnewline\hline
\endfirsthead
\multicolumn{3}{r}{...}
\endfoot
%\tabularnewline\hline
\endlastfoot
\hline
AliceBlue		&\#F0F8FF	 	&\cellcolor{AliceBlue}	\\
\hline
AntiqueWhite	&\#FAEBD7	 	&\cellcolor{AntiqueWhite}\\
\hline
Aqua		&\#00FFFF	 	&\cellcolor{Aqua}		\\
\hline
Aquamarine	&\#7FFFD4	 	&\cellcolor{Aquamarine}\\
\hline
Azure		&\#F0FFFF	 	&\cellcolor{Azure}		\\
\hline
Beige		&\#F5F5DC	 	&\cellcolor{Beige}		\\
\hline	
Bisque		&\#FFE4C4	 	&\cellcolor{Bisque}	\\
\hline
Black		&\#000000		&\cellcolor{Black}	 	\\
\hline
BlanchedAlmond&\#FFEBCD	 	&\cellcolor{BlanchedAlmond}\\
\hline
Blue			&\#0000FF	 	&\cellcolor{Blue}		\\
\hline
BlueViolet	&\#8A2BE2	 	&\cellcolor{BlueViolet}	\\
\hline
Brown		&\#A52A2A	 	&\cellcolor{Brown}	\\
\hline
BurlyWood	&\#DEB887		&\cellcolor{BurlyWood}\\
\hline
CadetBlue	&\#5F9EA0	 	&\cellcolor{CadetBlue}\\
\hline
Chartreuse	&\#7FFF00	 	&\cellcolor{Chartreuse}\\
\hline
Chocolate	&\#D2691E 		&\cellcolor{Chocolate}	 \\
\hline
Coral		&\#FF7F50	 	&\cellcolor{Coral}		\\
\hline
CornflowerBlue&\#6495ED	 	&\cellcolor{CornflowerBlue}\\
\hline
Cornsilk		&\#FFF8DC	 	&\cellcolor{Cornsilk}	\\
\hline
Crimson		&\#DC143C 		&\cellcolor{Crimson}	 \\
\hline
Cyan		&\#00FFFF	 &\cellcolor{Cyan}	\\
\hline
DarkBlue		&\#00008B&\cellcolor{DarkBlue}	 \\
\hline
DarkCyan		&\#008B8B&\cellcolor{DarkCyan}	 \\
\hline
DarkGoldenrod&\#B8860B	 &\cellcolor{DarkGoldenrod}\\
\hline
DarkGray		&\#A9A9A9	 &\cellcolor{DarkGray}\\
\hline
DarkGreen	&\#006400&\cellcolor{DarkGreen}	 \\
\hline
DarkKhaki	&\#BDB76B&\cellcolor{DarkKhaki}	 \\
\hline
DarkMagenta	&\#8B008B&\cellcolor{DarkMagenta}	 \\
\hline
DarkOliveGreen&\#556B2F	 &\cellcolor{DarkOliveGreen}\\
\hline
DarkOrange	&\#FF8C00	 &\cellcolor{DarkOrange}\\
\hline
DarkOrchid	&\#9932CC&\cellcolor{DarkOrchid}	 \\
\hline
DarkRed		&\#8B0000	 &\cellcolor{DarkRed}\\
\hline
DarkSalmon	&\#E9967A	 &\cellcolor{DarkSalmon}\\
\hline
DarkSeaGreen&\#8FBC8F	 &\cellcolor{DarkSeaGreen}\\
\hline
DarkSlateBlue&\#483D8B&\cellcolor{DarkSlateBlue}	 \\
\hline
DarkSlateGray&\#2F4F4F	 &\cellcolor{DarkSlateGray}\\
\hline
DarkTurquoise	&\#00CED1	 &\cellcolor{DarkTurquoise}\\
\hline
DarkViolet	&\#9400D3	 &\cellcolor{DarkViolet}\\
\hline
DeepPink		&\#FF1493	 &\cellcolor{DeepPink}\\
\hline
DeepSkyBlue	&\#00BFFF	 &\cellcolor{DeepSkyBlue}\\
\hline
DimGray		&\#696969	 &\cellcolor{DimGray}\\
\hline
DodgerBlue	&\#1E90FF	 &\cellcolor{DodgerBlue}\\
\hline
%Feldspar		&\#D19275	 &\cellcolor{Feldspar}\\
%\hline
FireBrick		&\#B22222	 &\cellcolor{FireBrick}\\
\hline
FloralWhite	&\#FFFAF0	 &\cellcolor{FloralWhite}\\
\hline
ForestGreen	&\#228B22&\cellcolor{ForestGreen}	 \\
\hline
Fuchsia		&\#FF00FF	 &\cellcolor{Fuchsia}\\
\hline
Gainsboro	&\#DCDCDC	 &\cellcolor{Gainsboro}\\
\hline
GhostWhite	&\#F8F8FF	 &\cellcolor{GhostWhite}\\
\hline
Gold		&\#FFD700	 &\cellcolor{Gold}\\
\hline
Goldenrod	&\#DAA520	 &\cellcolor{Goldenrod}\\
\hline
Gray			&\#808080	 &\cellcolor{Gray}	\\
\hline
Green		&\#008000&\cellcolor{Green}	 \\
\hline
GreenYellow	&\#ADFF2F	 &\cellcolor{GreenYellow}\\
\hline
Honeydew	&\#F0FFF0	 &\cellcolor{Honeydew}\\
\hline
HotPink		&\#FF69B4	 &\cellcolor{HotPink}\\
\hline
IndianRed		&\#CD5C5C&\cellcolor{IndianRed}	 \\
\hline
Indigo		&\#4B0082&\cellcolor{Indigo}	 	\\
\hline
Ivory		&\#FFFFF0	 &\cellcolor{Ivory}\\
\hline
Khaki		&\#F0E68C	 &\cellcolor{Khaki}\\
\hline
Lavender		&\#E6E6FA	 &\cellcolor{Lavender}\\
\hline
LavenderBlush&\#FFF0F5	 &\cellcolor{LavenderBlush}\\
\hline
LawnGreen	&\#7CFC00	 &\cellcolor{LawnGreen}\\
\hline
LemonChiffon	&\#FFFACD	 &\cellcolor{LemonChiffon}\\
\hline
LightBlue		&\#ADD8E6	 &\cellcolor{LightBlue}\\
\hline
LightCoral	&\#F08080	 &\cellcolor{LightCoral}\\
\hline
LightCyan	&\#E0FFFF	 &\cellcolor{LightCyan}\\
\hline
LightGoldenrodYellow&\#FAFAD2	 &\cellcolor{LightGoldenrodYellow}\\
\hline
LightGrey	&\#D3D3D3	 &\cellcolor{LightGrey}\\
\hline
LightGreen	&\#90EE90	 &\cellcolor{LightGreen}\\
\hline
LightPink		&\#FFB6C1	 &\cellcolor{LightPink}\\
\hline
LightSalmon	&\#FFA07A	 &\cellcolor{LightSalmon}	\\
\hline
LightSeaGreen&\#20B2AA	 &\cellcolor{LightSeaGreen}\\
\hline
LightSkyBlue	&\#87CEFA	 &\cellcolor{LightSkyBlue}\\
\hline
LightSlateBlue&\#8470FF	 &\cellcolor{LightSlateBlue}\\
\hline
LightSlateGray&\#778899&\cellcolor{LightSlateGray}	 \\
\hline
LightSteelBlue&\#B0C4DE&\cellcolor{LightSteelBlue}	 \\
\hline
LightYellow	&\#FFFFE0	 &\cellcolor{LightYellow}\\
\hline
Lime			&\#00FF00	 &\cellcolor{Lime}\\
\hline
LimeGreen	&\#32CD32	 &\cellcolor{LimeGreen}\\
\hline
Linen		&\#FAF0E6	 &\cellcolor{Linen}\\
\hline
Magenta		&\#FF00FF	 &\cellcolor{Magenta}\\
\hline
Maroon		&\#800000&\cellcolor{Maroon}	 \\
\hline
MediumAquamarine	&\#66CDAA	 &\cellcolor{MediumAquamarine}\\
\hline
MediumBlue	&\#0000CD	 &\cellcolor{MediumBlue}\\
\hline
MediumOrchid	&\#BA55D3	 &\cellcolor{MediumOrchid}\\
\hline
MediumPurple	&\#9370D8	 &\cellcolor{MediumPurple}\\
\hline
MediumSeaGreen	&\#3CB371&\cellcolor{MediumSeaGreen}	 \\
\hline
MediumSlateBlue	&\#7B68EE	 &\cellcolor{MediumSlateBlue}\\
\hline
MediumSpringGreen&\#00FA9A	 &\cellcolor{MediumSpringGreen}\\
\hline
MediumTurquoise	&\#48D1CC	 &\cellcolor{MediumTurquoise}\\
\hline
MediumVioletRed	&\#C71585	 &\cellcolor{MediumVioletRed}\\
\hline
MidnightBlue		&\#191970	 &\cellcolor{MidnightBlue}\\
\hline
MintCream		&\#F5FFFA	 &\cellcolor{MintCream}\\
\hline
MistyRose		&\#FFE4E1	 &\cellcolor{MistyRose}\\
\hline
Moccasin			&\#FFE4B5	 &\cellcolor{Moccasin}\\
\hline
NavajoWhite		&\#FFDEAD&\cellcolor{NavajoWhite}	 \\
\hline
Navy			&\#000080	 &\cellcolor{Navy}\\
\hline
OldLace			&\#FDF5E6	 &\cellcolor{OldLace}\\
\hline
Olive			&\#808000&\cellcolor{Olive}	 \\
\hline
OliveDrab		&\#6B8E23&\cellcolor{OliveDrab}	 \\
\hline
Orange			&\#FFA500	 &\cellcolor{Orange}\\
\hline
OrangeRed		&\#FF4500	&\cellcolor{OrangeRed} \\
\hline
Orchid			&\#DA70D6	 &\cellcolor{Orchid}\\
\hline
PaleGoldenrod	&\#EEE8AA	&\cellcolor{PaleGoldenrod} \\
\hline
PaleGreen		&\#98FB98	 &\cellcolor{PaleGreen}\\
\hline
PaleTurquoise		&\#AFEEEE	 &\cellcolor{PaleTurquoise}\\
\hline
PaleVioletRed		&\#D87093&\cellcolor{PaleVioletRed}	 \\
\hline
PapayaWhip		&\#FFEFD5	 &\cellcolor{PapayaWhip}\\
\hline
PeachPuff		&\#FFDAB9	 &\cellcolor{PeachPuff}\\
\hline
Peru				&\#CD853F	 &\cellcolor{Peru}\\
\hline
Pink				&\#FFC0CB	 &\cellcolor{Pink}\\
\hline
Plum			&\#DDA0DD&\cellcolor{Plum}	 \\
\hline
PowderBlue		&\#B0E0E6&\cellcolor{PowderBlue}	 \\
\hline
Purple			&\#800080	 &\cellcolor{Purple}\\
\hline
Red				&\#FF0000	 &\cellcolor{Red}\\
\hline
RosyBrown		&\#BC8F8F	 &\cellcolor{RosyBrown}\\
\hline
RoyalBlue		&\#4169E1	 &\cellcolor{RoyalBlue}\\
\hline
SaddleBrown		&\#8B4513&\cellcolor{SaddleBrown}	 \\
\hline
Salmon			&\#FA8072	 &\cellcolor{Salmon}\\
\hline
SandyBrown		&\#F4A460&\cellcolor{SandyBrown}	 \\
\hline
SeaGreen		&\#2E8B57	 &\cellcolor{SeaGreen}\\
\hline
Seashell			&\#FFF5EE	 &\cellcolor{Seashell}\\
\hline
Sienna			&\#A0522D&\cellcolor{Sienna}	 \\
\hline
Silver			&\#C0C0C0&\cellcolor{Silver}	 \\
\hline
SkyBlue			&\#87CEEB	 &\cellcolor{SkyBlue}\\
\hline
SlateBlue		&\#6A5ACD	&\cellcolor{SlateBlue} \\
\hline
SlateGray		&\#708090	 &\cellcolor{SlateGray}\\
\hline
Snow			&\#FFFAFA	 &\cellcolor{Snow}\\
\hline
SpringGreen		&\#00FF7F&\cellcolor{SpringGreen}	 \\
\hline
SteelBlue		&\#4682B4&\cellcolor{SteelBlue}	 \\
\hline
Tan				&\#D2B48C	 &\cellcolor{Tan}\\
\hline
Teal				&\#008080	 &\cellcolor{Teal}\\
\hline
Thistle			&\#D8BFD8&\cellcolor{Thistle}	 \\
\hline
Tomato			&\#FF6347	 &\cellcolor{Tomato}\\
\hline
Turquoise		&\#40E0D0&\cellcolor{Turquoise}	 \\
\hline
Violet			&\#EE82EE	 &\cellcolor{Violet}\\
\hline
VioletRed		&\#D02090&\cellcolor{VioletRed}	 \\
\hline
Wheat			&\#F5DEB3	 &\cellcolor{Wheat}\\
\hline
White			&\#FFFFFF	 &\cellcolor{White}\\
\hline
WhiteSmoke		&\#F5F5F5&\cellcolor{WhiteSmoke}	 \\
\hline
Yellow			&\#FFFF00	 &\cellcolor{Yellow}\\
\hline
YellowGreen		&\#9ACD32	&\cellcolor{YellowGreen}\\
\hline
\end{longtable}

























