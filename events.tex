\part{HTML Events}


HTML 4 的新特性之一是可以使 HTML 事件触发浏览器中的行为,比方说当用户点击某个 HTML 元素时启动一段 JavaScript。

在现代浏览器中都内置有大量的事件处理器。这些处理器会监视特定的条件或用户行为,例如鼠标单击或浏览器窗口中完成加载某个图像。通过使用客户端的 JavaScript,可以将某些特定的事件处理器作为属性添加给特定的标签,并可以在事件发生时执行一个或多个 JavaScript 命令或函数。

事件处理器的值是一个或一系列以分号隔开的 Javascript 表达式、方法和函数调用,并用引号引起来。当事件发生时,浏览器会执行这些代码。例如,当用户把鼠标移动到一个超链接时,会启动一个 JavaScript 函数。支持 JavaScript 的浏览器支持 <a> 标签中的一个特殊的 "mouse over"事件处理器 - 被称为 onmouseover 来完成这项工作:


\begin{lstlisting}[language=HTML]
     <a href="/index.html" onmouseover="alert('Welcome');return false"></a>
\end{lstlisting}

下面的表格提供了标准的事件属性,可以把它们插入 HTML/XHTML 元素中,以定义事件行为。





\chapter{Window Events}

仅在 body 和 frameset 元素中有效。


\begin{table}[!h]
\centering
\caption{窗口事件 (Window Events)}
\begin{tabular}{|l|l|l|}
\hline
属性		&值		&描述\\
\hline
onload	&脚本	&当文档被载入时执行脚本\\
\hline
onunload	&脚本	&当文档被卸下时执行脚本\\
\hline
\end{tabular}
\end{table}








\chapter{Form Element Events}

仅在表单元素中有效。


\begin{table}[!h]
\centering
\caption{表单元素事件 (Form Element Events)}
\begin{tabular}{|l|l|l|}
\hline
属性		&值		&描述\\
\hline
onchange&脚本	&当元素改变时执行脚本\\
\hline
onsubmit	&脚本	&当表单被提交时执行脚本\\
\hline
onreset	&脚本	&当表单被重置时执行脚本\\
\hline
onselect	&脚本	&当元素被选取时执行脚本\\
\hline
onblur	&脚本	&当元素失去焦点时执行脚本\\
\hline
onfocus	&脚本	&当元素获得焦点时执行脚本\\
\hline
\end{tabular}
\end{table}








\chapter{Image Events}

该属性可用于 img 元素:

\begin{table}[!h]
\centering
\caption{图像事件 (Image Events)}
\begin{tabular}{|l|l|l|}
\hline
属性		&值		&描述\\
\hline
onabort	&脚本	&当图像加载中断时执行脚本\\
\hline
\end{tabular}
\end{table}




\chapter{Keyboard Events}


在下列元素中无效:base、bdo、br、frame、frameset、head、html、iframe、meta、param、script、style 以及 title 元素。

\begin{table}[!h]
\centering
\caption{键盘事件 (Keyboard Events)}
\begin{tabular}{|l|l|l|}
\hline
属性			&值	&描述\\
\hline
onkeydown	&脚本	&当键盘被按下时执行脚本\\
\hline
onkeypress	&脚本	&当键盘被按下后又松开时执行脚本\\
\hline
onkeyup		&脚本	&当键盘被松开时执行脚本\\
\hline
\end{tabular}
\end{table}







\chapter{Mouse Events}

在下列元素中无效:base、bdo、br、frame、frameset、head、html、iframe、meta、param、script、style 以及 title 元素。

\begin{table}[!h]
\centering
\caption{鼠标事件 (Mouse Events)}
\begin{tabular}{|l|l|l|}
\hline
属性			&值		&描述\\
\hline
onclick		&脚本	&当鼠标被单击时执行脚本\\
\hline
ondblclick	&脚本	&当鼠标被双击时执行脚本\\
\hline
onmousedown	&脚本	&当鼠标按钮被按下时执行脚本\\
\hline
onmousemove	&脚本	&当鼠标指针移动时执行脚本\\
\hline
onmouseout	&脚本	&当鼠标指针移出某元素时执行脚本\\
\hline
onmouseover	&脚本	&当鼠标指针悬停于某元素之上时执行脚本\\
\hline
onmouseup	&脚本	&当鼠标按钮被松开时执行脚本\\
\hline
\end{tabular}
\end{table}

















